\documentclass{article}
\usepackage[spanish]{babel}
\usepackage[utf8]{inputenc}
\usepackage{graphicx}
\usepackage{hyperref}

\title{Informe del Proyecto de Realidad Aumentada Educativa sobre Extintores}
\author{Matias Bozo, Felipe Salazar y Bryan Pheng}
\date{\today}

\begin{document}

\maketitle

\begin{abstract}
Este informe detalla el desarrollo de una aplicación de realidad aumentada diseñada para enseñar a los usuarios sobre los diferentes tipos de extintores y sus usos apropiados. La aplicación combina elementos interactivos y educativos para crear una experiencia de aprendizaje dinámica y atractiva.
\end{abstract}

\section{Introducción}
La realidad aumentada (RA) ofrece nuevas posibilidades para la educación. Este proyecto explora el uso de la RA para enseñar sobre la seguridad contra incendios y el uso correcto de los extintores. La aplicación utiliza marcadores de RA para mostrar información sobre diferentes tipos de extintores cuando se escanean los códigos de barras correspondientes.

\section{Objetivos}
El objetivo principal de este proyecto es proporcionar una herramienta educativa que:
\begin{itemize}
  \item Enseñe a los usuarios sobre los diferentes tipos de extintores.
  \item Demuestre el uso adecuado de cada tipo de extintor en diferentes situaciones de fuego.
  \item Ofrezca una experiencia de aprendizaje interactiva y dinámica a través de la RA.
\end{itemize}

\section{Metodología}
La aplicación fue diseñada y desarrollada utilizando A-Frame y AR.js. Se implementaron marcadores de RA para los extintores, y se programó la lógica para mostrar información sobre los extintores cuando se escanean los códigos de barras correspondientes.

\section{Diseño de la Aplicación}
La interfaz de usuario y la experiencia del usuario fueron diseñadas para ser intuitivas y educativas. Aunque la aplicación no tiene una interfaz de usuario fuera de A-Frame, se ha hecho un esfuerzo para que la interacción con los marcadores de RA sea lo más sencilla posible.

\section{Implementación}
Los marcadores de RA se implementaron utilizando A-Frame y AR.js. Se programó la lógica para mostrar información sobre los extintores cuando se escanean los códigos de barras correspondientes.

\section{Resultados}
La aplicación cumple con los objetivos establecidos y ha demostrado ser eficaz en un entorno educativo. Los usuarios han informado que la aplicación es útil para aprender sobre los diferentes tipos de extintores y cómo usarlos correctamente.

\section{Conclusiones}
Este proyecto ha demostrado que la RA puede ser una herramienta eficaz para la educación. La aplicación ha tenido un impacto positivo en la educación sobre seguridad contra incendios, y se espera que continúe siendo una valiosa herramienta educativa en el futuro.

\section{Trabajo Futuro}
Hay varias posibles mejoras o expansiones para este proyecto. Estas incluyen agregar más elementos interactivos, expandir el contenido educativo, y explorar otras aplicaciones de la RA en la educación.

\end{document}